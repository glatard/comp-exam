\chapter{Reduced precision}
\begin{comment}
This section will discuss:
1. What is reduced precision (RP); discuss mixed precision as a sub-category.
2. Advantages of RP (Can make mention on energy consumption)
3. Disavantages of RP.
4. Current field of application (mostly DL).

Alternative outline:
This section will discuss:
1. What is reduced precision (RP); discuss mixed precision as a sub-category.
2. Examples of RP application.
3. Discussion of the prior examples + advanatages/disavantages.
4. Current field of application (mostly DL).

3rd alternative (Same as section 2 but written historically):
historical outline of RP.
How it started, and how it exploded with DL.
RP is great. How relevant to other fields?
\end{comment}

% Make the mention of mixed precision and how the use of RP will englobe both 
% concepts for the remaining of the text.




\section{Data format for reduced precision}
% pytorch and tensorflow data type for mixed precision

% \section{State of reduced precision}


% Application in DL and other fields
% Also, discuss the scope of application
\section{Growth of reduced-precision in Deep Learning}

\section{Other fields of application}

\section{Current scope of application}


\section{Tradeoff of using reduced precision}
% Discuss the main advantages and disavantage of using RP
% For example perfomance, overhead, developer effort, energy cost, etc.
% Reuse the above example to describe those pros/cons

% \section{Adoption rate}
% Talk about this through the other sections.
